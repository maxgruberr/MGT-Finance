\documentclass{article}
\usepackage{graphicx}
\usepackage{amsmath}
\usepackage{float}
\usepackage{geometry}
\geometry{a4paper, margin=1in}

\title{Homework 2, Part 1 Solutions}
\author{Your Name}
\date{\today}

\begin{document}

\maketitle

% --- INSTRUCTIONS ---
% 1. Copy the full content of this file into a new project on Overleaf.
% 2. Upload the 'efficient_frontier.png' file generated by the Python script to your Overleaf project.
% 3. Run the 'hw2_solutions.py' script.
% 4. Copy the numerical values from the script's output and paste them into the corresponding [PLACEHOLDER] fields below.
% 5. Compile the LaTeX document to generate the PDF with your answers.

\section*{Question 4: Tangency Portfolio}
\label{sec:tangency}

This section details the composition and performance metrics of the tangency portfolio, which is constructed from the three risky ETFs: EWL, IEF, and SPY.

\subsection*{Portfolio Weights}
The weights of the three risky ETFs in the tangency portfolio are as follows:
\begin{itemize}
    \item \textbf{EWL Weight:} [PASTE TANGENCY WEIGHT FOR EWL HERE] \%
    \item \textbf{IEF Weight:} [PASTE TANGENCY WEIGHT FOR IEF HERE] \%
    \item \textbf{SPY Weight:} [PASTE TANGENCY WEIGHT FOR SPY HERE] \%
\end{itemize}

\subsection*{Performance Metrics}
The key performance metrics for the tangency portfolio are:
\begin{itemize}
    \item \textbf{Mean (Annualized Return):} [PASTE MEAN RETURN HERE] \%
    \item \textbf{Variance:} [PASTE VARIANCE HERE]
    \item \textbf{Standard Deviation (Annualized Volatility):} [PASTE STANDARD DEVIATION HERE] \%
    \item \textbf{Sharpe Ratio:} [PASTE SHARPE RATIO HERE]
\end{itemize}

\section*{Question 5: Mean-Standard Deviation Frontier}
\label{sec:frontier}

The plot below illustrates the mean-standard deviation frontier for portfolios constructed from the three risky ETFs. It also includes the Capital Allocation Line (CAL), which represents the risk-return combinations achievable by mixing the tangency portfolio with the risk-free asset.

\begin{figure}[H]
    \centering
    % Ensure 'efficient_frontier.png' is uploaded to your Overleaf project directory
    \includegraphics[width=0.8\textwidth]{efficient_frontier.png}
    \caption{The Mean-Standard Deviation Frontier and the Capital Allocation Line.}
    \label{fig:frontier}
\end{figure}

\section*{Question 6: Portfolio Optimization for Target Volatility}
\label{sec:optimization}

This section presents the optimal portfolio allocation required to achieve a target annual volatility of 15\%.

\subsection*{Optimal Weights for 15\% Volatility}
To achieve a target annual volatility of 15\%, the optimal weights in the three risky ETFs are:
\begin{itemize}
    \item \textbf{Optimal EWL Weight:} [PASTE OPTIMAL WEIGHT FOR EWL HERE] \%
    \item \textbf{Optimal IEF Weight:} [PASTE OPTIMAL WEIGHT FOR IEF HERE] \%
    \item \textbf{Optimal SPY Weight:} [PASTE OPTIMAL WEIGHT FOR SPY HERE] \%
\end{itemize}

\subsection*{Implied Risk-Aversion Coefficient}
The implied risk-aversion coefficient for an investor choosing this portfolio is:
\begin{itemize}
    \item \textbf{Risk-Aversion Coefficient:} [PASTE RISK-AVERSION COEFFICIENT HERE]
\end{itemize}

\end{document}
